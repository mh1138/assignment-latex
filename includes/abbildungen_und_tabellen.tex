\chapter{Beispiele für Abbildungen und Tabellen}\label{chapter:abbildungenTabellen}

Hier finden Sie Beispiele für Abbildungen, Tabellen, Formelsatz und  Source Code.

\section{Abbildungen}
In diesem Abschnitt gibt die Abbildungen~\ref{abb:Logo2cmHoch} und~\ref{abb:Logo2cmBreit}, die beide das Logo der DHBW zeigen.

\begin{figure}[htb]
\centering
\includegraphics[height=2cm]{graphics/dhbw.png}
\caption[DHBW-Logo 2cm hoch]{DHBW-Logo 2cm hoch.\footnotemark}
\label{abb:Logo2cmHoch}
\end{figure}
\footnotetext{Mit Änderungen entnommen aus: \cite{onl:DHBW}}

\lstset{language=TeX, % hervorzuhebende Keywords definieren
  morekeywords={footnotetext,footnotemark,footcite,caption}
}

\emph{Spezialfall:} Sofern \emph{innerhalb} der Bezeichnung einer Abbildung eine Fußnote angegeben oder eine Quelle referenziert werden soll, geschieht dies nicht per \lstinline|\footnote| oder \lstinline
|\footcite|. Vielmehr sind die Befehle \lstinline|\footnotemark| und \lstinline|\footnotetext| zu verwenden und außerdem das optionale Argument für \lstinline|\caption| anzugeben (vgl.\ Source Code).

\begin{figure}[htb]
\centering
\includegraphics[width=2cm]{graphics/dhbw.png}
\caption[DHBW-Logo 2cm breit.]{DHBW-Logo 2cm breit.\footnotemark}
\label{abb:Logo2cmBreit}
\end{figure}
\footnotetext{dito}



\section[Tabellen]{Tabellen\footnote{Erklärungen von Prof. Dr. Brandt.}}\label{sec:tabellen}

In diesem Abschnitt finden Sie zwei unterschiedlich große Beispiel-Tabellen (Tabellen \ref{tab:BeispielTabelleKleiner} und  \ref{tab:BeispielTabelleGroesser}), welche den Regeln wichtiger Verlage entsprechen. Für die Erstellung von Tabellen empfiehlt sich das Paket \texttt{booktabs}, welches sich in der Handhabung kaum von den Standardbefehlen unterscheidet, jedoch wesentlich bessere Ergebnisse liefert.

\begin{table}[h]
  \centering
  \ra{1.3}
  \begin{tabular}{@{}llllll@{}}
      \toprule
      {Spalte 1} & {Spalte 2} & {Spalte 3} & {Spalte 4} & {Spalte 5} & {Spalte 6} \\
      \midrule
      a        & b          & c                & d        & e        & f        \\
      Test     & Test, Test & Test, Test, Test & ~        & ~        & ~        \\
      1        & 2          & 3                & 4        & 5        & 6        \\
      \bottomrule
  \end{tabular}
  \caption{Kleinere Beispiel-Tabelle.}
  \label{tab:BeispielTabelleKleiner}
\end{table}

Folgende Stilregeln sollten bei der Erstellung von Tabellen beachtet werden:
\begin{itemize}
\item Vermeiden Sie vertikale Linien.
\item Drei horizontale Linien reichen aus: eine über der Tabelle (\texttt{\textbackslash toprule}), eine unter der Tabelle (\texttt{\textbackslash bottomrule}) und eine, die den Tabellenkopf vom Rest der Tabelle trennt (\texttt{\textbackslash midrule}).
\item Horizontaler Abstand zwischen den Zeilen erhöht die Lesbarkeit (hier mit \texttt{booktabs} durch \texttt{\textbackslash ra\{1.3\}} realisiert, wobei zuvor in der Präambel \texttt{\textbackslash newcommand\{\textbackslash ra\}[1]\{\textbackslash renewcommand\{ \textbackslash arraystretch\}\{\#1\}\}} definiert wurde).

\item Falls Sie mehrstufige Überschriften verwenden, benötigen Sie möglicherweise mehrere Linien. Horizontale Linien über eine Teilmenge von Spalten können mit \texttt{\textbackslash cmidrule} erzeugt werden. Mit dem optionalen Argument \texttt{\textbackslash cmidrule(lr)} wird die Linie am Anfang und Ende etwas beschnitten.
\item Mehrstufige Zeilenbeschriftungen sollten durch Zwischenüberschriften und kleine Zusatzabstände realisiert werden (hier mit \texttt{\textbackslash addlinespace}).
\item Verwenden Sie niemals doppelte Linien.
\item Sie benötigen an den vertikalen Kanten der Tabelle keine Abstände (d.h.\ vor der ersten und nach der letzten Spalte wird \texttt{@\{\}} im Argument von \texttt{\textbackslash begin\{tabular\}} angegeben).
\item Achten Sie auf Einheiten, Tausendertrennzeichen und Dezimalzeichen.
\end{itemize}

\begin{table}[t]
  \centering
  \ra{1.3} 
  \begin{tabular}{@{}rrrrr@{}}
  \toprule
  & \multicolumn{2}{c}{{mean (\$)}} 
  & \multicolumn{2}{c}{{std (\$)}} \\
  \cmidrule(lr){2-3} \cmidrule(lr){4-5}
  {aspiration} 
  & {diesel} 
  & {gas} 
  & {diesel} 
  & {gas} \\
  \midrule
  \textit{front-engine}\\
  standard   & 9,670.57  & 12,246.34 & 4,004.01 & 7,753.99 \\
  turbo & 19,159.15 & 14,613.22 & 7,292.95 & 4,892.87 \\ \addlinespace
  \textit{rear-engine}\\
  standard    & \textemdash & 34,528.00 & \textemdash & 2,291.29 \\
  \bottomrule
  \end{tabular}
  \caption{Tabelle mit mehrstufigen Zeilen- und Spaltenbeschriftungen.}
  \label{tab:BeispielTabelleGroesser}
\end{table}

Die wichtigsten \emph{Platzierungsoptionen} für Tabellen (und Abbildungen) im Einzelnen:
 \begin{itemize}
 \item \texttt{h} -- here: möglichst genau an der Stelle, an der der Befehl im Quelltext steht
 \item \texttt{t} -- top: am Seitenanfang
 \item \texttt{b} -- bottom: am Seitenende
 \end{itemize}



\section{Etwas Mathematik}

Eine abgesetzte Formel:
\[
  \int_a^b x^2 \: \mathrm{d} x = \frac{1}{3} (b^3 - a^3)
\]

Es ist $a^2+b^2 = c^2$ eine Formel im Text.

Falls Ihr Dokument in größerem Umfang mathematische Formeln enthält, sollten Sie das Paket \texttt{amsmath}\footnote{verfügbar unter: \url{https://www.ctan.org/pkg/amsmath}} einbinden. Damit lassen sich zum Beispiel nummerierte Gleichungen und Matrizen darstellen wie in folgendem Beispiel:

\begin{equation}
A=
\begin{pmatrix}  
a_{11} & a_{12} \\  
a_{21} & a_{22}  
\end{pmatrix}
\end{equation}

\section{Source Code}

Source Code-Blöcke können auf folgende Arten eingefügt werden:

\lstset{language=Java}

Direkt im \LaTeX-Source Code:
\begin{lstlisting}
if(1 > 0) {
  System.out.println("OK"); 
} else {
  System.out.println("merkwuerdig");
}
\end{lstlisting}

oder eingefügt aus einer externen Datei.
\lstinputlisting{includes/HelloWorld.java}